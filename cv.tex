% English CV for Victor Hallberg

\def\name{Victor Hallberg}
\date{2013-02-17}
\title{Curriculum Vitae}
\author{\name}

%{{{ Document setup
\documentclass[a4paper,11pt]{article}

% Text configuration
\usepackage[utf8]{inputenc}
\usepackage[swedish]{babel}
\usepackage{lmodern} % font
%\renewcommand{\familydefault}{\sfdefault} % default family

% Additional symbols
\usepackage{amssymb}
% [http://www.tex.ac.uk/tex-archive/info/symbols/comprehensive/symbols-a4.pdf]
\usepackage{marvosym}
\usepackage{textcomp}

% Colors [http://ctan.uib.no/macros/latex/contrib/xcolor/xcolor.pdf]
\usepackage[svgnames]{xcolor}

% Paragraphs
\parindent 0mm
\parskip 0mm

% Margins
\usepackage[top=1.6cm, bottom=1.6cm, left=2.5cm, right=2cm]{geometry}
\newlength{\titleleft}
\newlength{\titlefwidth}
\setlength{\titlefwidth}{\textwidth}
\setlength{\titleleft}{-0.5cm}
\addtolength{\titlefwidth}{0.5cm}

% Header and text formatting
\usepackage{titlesec}

% CV sections
\titleformat{\section}{\sf \bfseries \Large}{\thesection}{}{}[\vspace{-5mm}\rule{\titlefwidth}{1pt}]
% Work places / bold leaders
\titleformat{\subsection}[runin]{\normalsize}{\thesubsection}{}{}
% Work titles
\titleformat{\subsubsection}{\normalsize}{\thesubsubsection}{}{}

% Spacing
\titlespacing{\section}{\titleleft}{5mm}{1mm}
\titlespacing{\subsection}{\titleleft}{3mm}{1mm}
\titlespacing{\subsubsection}{1mm}{2mm}{1mm}

% Hyperlinking
\usepackage[pdftex]{hyperref}
\hypersetup{%
	colorlinks = true,%
	linkcolor  = Black,%
	urlcolor   = Navy%
}

% Footer
\usepackage{lastpage}
\usepackage{fancyhdr}
\pagestyle{fancy}
\renewcommand{\headrulewidth}{0pt}
\cfoot{\thepage\ / \pageref{LastPage}}

%}}}
%{{{ Custom formats and commands

\newcommand{\icon}[1]{\textcolor{lightgray}{#1}}
\newcommand{\iconl}[1]{\hspace{-0.5cm}\makebox[0.3cm][c]{\icon{#1}}\hspace{0.2cm}}

% Headers
\newcommand{\theheader}[0]{\sf \LARGE \bfseries \name}
\newcommand{\worktitle}[1]{\textbf{#1}}
\newcommand{\duration}[1]{\textsl{#1}}
\newcommand{\theplace}[2][]{\subsection*{\textbf{#2}#1}}
\newcommand{\thework}[2]{\subsubsection*{%
  \iconl{$\blacksquare$}%
  \worktitle{#1}%
  \hfill\duration{#2}}\par%
}
\newcommand{\thebold}[1]{\subsection*{\textbf{#1}}\vspace{-1mm}}
%}}}
%{{{ Keywords
\def\html{\href{http://en.wikipedia.org/wiki/HTML}{HTML}}
\def\javascript{\href{http://en.wikipedia.org/wiki/JavaScript}{JavaScript}}
\def\css{\href{http://en.wikipedia.org/wiki/Cascading_Style_Sheets}{CSS}}
\def\clang{\href{http://en.wikipedia.org/wiki/C_\%28programming_language\%29}{C}}
\def\regex{\href{http://en.wikipedia.org/wiki/Regular_expressions}{RegEx}}
\def\git{\href{http://git-scm.com/}{Git}}
\def\svn{\href{http://subversion.apache.org/}{Subversion}}
\def\jquery{\href{http://jquery.com/}{jQuery}}
\def\wordpress{\href{http://wordpress.org}{WordPress}}
\def\ror{\href{http://rubyonrails.org}{Ruby on Rails}}
\def\java{\href{http://www.java.com/}{Java}}
\def\php{\href{http://php.net}{PHP}}
\def\ruby{\href{http://www.ruby-lang.org/}{Ruby}}
\def\csharp{\href{http://en.wikipedia.org/wiki/C_Sharp_\%28programming_language\%29}{C\#.net}}
\def\python{\href{http://www.python.org/}{Python}}
\def\mysql{\href{http://www.mysql.com/}{MySQL}}
\def\apache{\href{http://httpd.apache.org/}{Apache}}
\def\iis{\href{http://www.iis.net/}{IIS}}
\def\solr{\href{http://lucene.apache.org/solr/}{Solr}}
\def\eclipse{\href{http://www.eclipse.org/eclipse/}{Eclipse}}
\def\vs{\href{http://www.microsoft.com/visualstudio/}{Visual Studio}}
\def\vim{\href{http://www.vim.org/}{vim}}
\def\photoshop{\href{http://www.adobe.com/products/photoshop.html}{Photoshop}}
\def\office{\href{http://office.microsoft.com/}{Microsoft Office}}
\def\latex{\href{http://en.wikipedia.org/wiki/LaTeX}{LaTeX}}
%}}}

\begin{document}

%{{{ Header

% Name
\hspace{\titleleft}\parbox{\titlefwidth}{%
	{\theheader}\vspace{-2mm}\par%
	\rule{\titlefwidth}{1pt}\vspace{2mm} % ruler
}

% Address information
\begin{tabular*}{\textwidth}{@{}l l @{\extracolsep{\fill}} r @{}}
\iconl{\bf \Large \textborn} {\bf Date of birth:} & 21 January 1989     & Forskarbacken 19, lgh 1102\\
\iconl{\Telefon} {\bf Phone number:}  & +46 (0)70-283 55 65 & 114 15 Stockholm, Sweden \\
\iconl{\Letter} {\bf Email address:} & \href{mailto:victorha@kth.se}{\tt victorha@kth.se} &
\end{tabular*}

%}}}
\section*{Education}%{{{
\theplace[, Stockholm, Sweden]{\href{http://kth.se}{Kungliga Tekniska Högskolan (The Royal Institute of Technology)}}

\thework{Master of Science degree in Computer Science}{2011.08--present}
Ongoing M.Sc. degree with an anticipated graduation in December 2013.

\thework{Bachelor of Science degree in Computer Science}{2008.08--2011.06}
\vspace{4mm}

%}}}
\section*{Experience}%{{{
\theplace[, Stockholm, Sweden]{\href{http://www.netgaming.se}{Net Gaming Europe AB}}

\thework{Web developer intern}{2013.01--present}
Was responsible for developing the frontend for a \href{http://casinoloco.com}{online casino site} launched during the three month internship. The site used \wordpress\ as a platform and the intership involved design work, custom feature implementation and development of a completely responsive casino site accessible by a multitude of devices.

\theplace[, Stockholm, Sweden]{\href{http://www.polisforbundet.se}{Polisförbundet (The Swedish Police Union)}}

\thework{Web developer (contract)}{2011.03--present}
Developed and launched a \wordpress-based web site for \href{http://www.polistidningen.se}{Polistidningen}, a magazine for members of the \mbox{Swedish} Police Union. The work consisted of implementing a web design by specification along with developing a number of custom functionalities specific to the site. I have also maintained and configured their web servers (both Windows and Linux environments) since the launch.

\theplace[, Stockholm, Sweden]{\href{http://www.sverigesingenjorer.se}{Sveriges Ingenjörer (The Swedish Association of Graduate Engineers)}}

\thework{Web developer and designer (contract)}{2009.09--2012.05}
Developed a new version of \href{http://www.ingenjoren.se}{Ingenjören}, a regularly published magazine for members of Sveriges Ingenjörer. I have since the launch in 2009 provided support and developed several additions for the site.

\thework{Web developer and designer (contract)}{2010.05--2010.12}
Designed and implemented a new design for \href{http://www.ingenjorsbloggen.se}{Ingenjörsbloggen}, the official blog of Sveriges Ingenjörer. The project, developed as a \wordpress\ theme, also included implementation of social network functionalities.

\theplace[, Solna, Sweden]{Siemens IT Solutions and Services}

\thework{User and systems tester}{2008.06--2008.07}
Worked as a tester on a SAP system being developed for a large company during the summer break.

%}}}
\section*{Technical expertise}%{{{
I have a broad knowledge in and experience with programming in general but specialize in web development, primarily on the front end (interface and design implementation).

\thebold{Languages:} \html, \css, \javascript/\jquery, \java, \ruby, \php, \clang(++), basic \csharp\ and \python.

\thebold{Tools:} \wordpress, \ror, \regex, \git, \svn, \latex\ and shell scripting.

\thebold{Software:} \mysql, \apache, \iis, \solr, \eclipse, \vs, \vim, \photoshop\ and \office.

\thebold{Operating systems:} Microsoft Windows, Unix/Linux and Mac OS X.
%}}}
\section*{Languages}%{{{
\textbf{Swedish} is my native language. I also speak and write \textbf{English} on an academic level.
%}}}
\section*{Extracurricular  experience}%{{{
I have been active in a number of organisations and groups during my time at the university.

\theplace[, Stockholm, Sweden]{\href{http://kth.se}{Kungliga Tekniska Högskolan (The Royal Institute of Technology)}}

\thework{Study tour organizer}{2012.07--present}
I'm head of IT and communications in the \href{http://studieresan.se}{StuDs 2013} project, the study tour for members of the computer science chapter. The goal of the project includes meeting and connecting with interesting companies in Sweden and other countries which we will be visiting during the end of the course.

\thework{30-year anniversary organizer}{2012.05--present}
I organize the \href{http://djubileet.se/}{30th year anniversary} of the computer science chapter together with a group of five other people. The anniversary involves multiple events of different sizes (each one enganging between 40 and 700 people) leading up to the 30th birthday of the chapter.

\thework{Mentor for new students}{2009, 2010, 2011, 2012}
I have, since my second year at KTH, worked as a mentor for the new computer science students arriving each year. This involves elaborate planning and full time work during the first three weeks of the semester.
%}}}

\end{document}

