% Swedish CV for Victor Hallberg

\def\name{Victor Hallberg}
\date{2012-02-10}
\title{Curriculum Vitae}
\author{\name}

% {{{ Document setup
\documentclass[a4paper,10pt]{article}

% Text configuration
\usepackage[utf8]{inputenc}
\usepackage[swedish]{babel}
\usepackage{lmodern} % font
%\renewcommand{\familydefault}{\sfdefault} % default family

% Additional symbols
\usepackage{amssymb}
% [http://www.tex.ac.uk/tex-archive/info/symbols/comprehensive/symbols-a4.pdf]
\usepackage{marvosym}
\usepackage{textcomp}

% Colors [http://ctan.uib.no/macros/latex/contrib/xcolor/xcolor.pdf]
\usepackage[svgnames]{xcolor}

% Paragraphs
\parindent 0mm
\parskip 0mm

% Margins
\usepackage[top=1.8cm, bottom=1.8cm, left=2.5cm, right=2cm]{geometry}
\newlength{\titleleft}
\newlength{\titlefwidth}
\setlength{\titlefwidth}{\textwidth}
\setlength{\titleleft}{-0.5cm}
\addtolength{\titlefwidth}{0.5cm}

% Header and text formatting
\usepackage{titlesec}

% CV sections
\titleformat{\section}{\sf \bfseries \Large}{\thesection}{}{}[\vspace{-5mm}\rule{\titlefwidth}{1pt}]
% Work places / bold leaders
\titleformat{\subsection}[runin]{\normalsize}{\thesubsection}{}{}
% Work titles
\titleformat{\subsubsection}{\normalsize}{\thesubsubsection}{}{}

% Spacing
\titlespacing{\section}{\titleleft}{5mm}{1mm}
\titlespacing{\subsection}{\titleleft}{3mm}{1mm}
\titlespacing{\subsubsection}{0cm}{2mm}{1mm}

% Custom formatters
\newcommand{\icon}[1]{\textcolor{lightgray}{#1}}
\newcommand{\iconl}[1]{\hspace{-0.5cm}\makebox[0.3cm][c]{\icon{#1}}\hspace{0.2cm}}

% Custom headers
\newcommand{\theheader}[0]{\sf \LARGE \bfseries \name}
\newcommand{\worktitle}[1]{\textbf{#1}}
\newcommand{\duration}[1]{\textsl{#1}}
\newcommand{\theplace}[2][]{\subsection*{\textbf{#2}#1}}
\newcommand{\thework}[2]{\subsubsection*{%
  \iconl{$\blacksquare$}%
  \worktitle{#1}%
  \hfill\duration{#2}}\par%
}
\newcommand{\thebold}[1]{\subsection*{\textbf{#1}}\vspace{-1mm}}

% Hyperlinking
\usepackage[pdftex]{hyperref}
\hypersetup{
	colorlinks = true,
	linkcolor  = NavyBlue,
	urlcolor   = NavyBlue
}

% Footer
\usepackage{lastpage}
\usepackage{fancyhdr}
\pagestyle{fancy}
\renewcommand{\headrulewidth}{0pt}
\cfoot{\thepage\ / \pageref{LastPage}}
% }}}

\begin{document}

%{{{ Header
% Name
\hspace{\titleleft}\parbox{\titlefwidth}{
	{\theheader}\vspace{-2mm}\par
	\rule{\titlefwidth}{1pt}\vspace{2mm} % ruler
}

% Address information
\begin{tabular*}{\textwidth}{@{}l l @{\extracolsep{\fill}} r @{}}
\iconl{\bf \Large \textborn} {\bf Date of birth:} & 21 January 1989     & Forskarbacken 19, lgh 1102\\
\iconl{\Telefon} {\bf Phone number:}  & +46 (0)70-283 55 65 & 114 15 Stockholm, Sweden \\
\iconl{\Letter} {\bf Email address:} & \href{mailto:victorha@kth.se}{\tt victorha@kth.se} &
\end{tabular*}
%}}}
\section*{Education}%{{{
\theplace[, Stockholm, Sweden]{\href{http://kth.se}{Kungliga Tekniska Högskolan (The Royal Institute of Technology)}}

\thework{Master of Science degree in Computer Science}{2008.08--present}
Completed bachelor's degree and ongoing master's degree in computer science.

\theplace[, Täby, Sweden]{\href{http://www.taby.se/ava}{Åva Gymnasium}}

\thework{Technological education program}{2005.08--2008.06}
Three years of study with a math and computer science profile in gymnasium (upper secondary school).
%}}}
\section*{Experience}%{{{
\theplace[, Stockholm, Sweden]{\href{http://www.polisforbundet.se}{Polisförbundet (The Swedish Police Union)}}

\thework{Web developer}{2011.03--present}
Developed and launched a WordPress-based web site for \href{http://www.polistidningen.se}{Polistidningen}, a magazine for members of the Swedish Police Union. The work consisted of implementing a web design by specification along with developing a number of custom functionalities specific to the site. I have also maintained and configured their web servers (in both Windows and Linux environments) since the launch.

\theplace[, Stockholm, Sweden]{\href{http://www.sverigesingenjorer.se}{Sveriges Ingenjörer (The Swedish Association of Graduate Engineers)}}

\thework{Web developer and designer}{2010.05--2010.12}
Designed and implemented a new design for \href{http://www.ingenjorsbloggen.se}{Ingenjörsbloggen}, the official blog of Sveriges Ingenjörer. The project, developed as a WordPress theme, also included implementation of social network functionalities.

\thework{Web developer and designer}{2009.09--present}
In 2009 I developed and launched a new WordPress-based web magazine for \href{http://www.ingenjoren.se}{Ingenjören}, a regularly published magazine for members of Sveriges Ingenjörer. I have since then provided support and developed several additions for the site.

\theplace[, Solna, Sweden]{Siemens IT Solutions and Services}

\thework{User and systems tester}{2008.06--2008.07}
Worked as a tester on a SAP system being developed for a large company during the summer break.

\theplace[, Upplands Väsby, Sweden]{Siemens Power Generation}

\thework{Server/System maintenace}{2005--2007}
Maintained and installed various servers and workstations during the holidays.
%}}}
\section*{Technical expertise}%{{{
I have a broad knowledge in and experience with programming in general but specialize in web development, primarily on the front end (interface and design implementation).

\thebold{Languages:} HTML, CSS, JavaScript/jQuery, Java, Ruby, PHP, C(++), basic C\#.net and Python.

\thebold{Tools:} WordPress, Ruby on Rails, Regular Expressions, Git, Subversion, Shell scripting and LaTeX.

\thebold{Software:} MySQL, Apache, Microsoft IIS, Eclipse, Visual Studio, vim, Adobe Photoshop and Microsoft Office.

\thebold{Operating systems:} Microsoft Windows, Unix/Linux and Mac OS X.
%}}}
\section*{Languages}%{{{
\textbf{Swedish} is my native language. I also speak and write \textbf{English} on an academical level.
%}}}
\section*{Extracurricular  experience}%{{{
I have been active in a number of organisations and groups during my time studying at the university.

\theplace[, Stockholm, Sweden]{\href{http://kth.se}{Kungliga Tekniska Högskolan (The Royal Institute of Technology)}}

\thework{30-year anniversary organizer}{2012.05--present}
I organize the 30th year anniversary of the computer science chapter together with a group of five other people. The anniversary involves multiple events of different sizes (each one enganging between 40 and 700 people) leading up to the 30th birthday of the chapter.

\thework{Mentor for new students}{2009, 2010, 2011, 2012}
I have, since my second year at KTH, worked as a mentor for the new computer science students arriving each year. This involves elaborate planning and full time work during the first three weeks of the semester.
%}}}
\end{document}

